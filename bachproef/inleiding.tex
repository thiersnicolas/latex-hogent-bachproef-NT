%%=============================================================================
%% Inleiding
%%=============================================================================

\chapter{\IfLanguageName{dutch}{Inleiding}{Introduction}}%
\label{ch:inleiding}

Wanneer nieuwe applicaties worden opgezet, die bepaalde functionaliteiten ter beschikking zullen stellen van andere applicaties, zal er bij aanvang gekeken worden
op welke manier de communicatie met die applicaties zal verlopen. Er moet dan voor een specifiek API gekozen worden.
API staat voor Application programming interfaces en is een software-interface die het mogelijk maakt voor twee
applicaties om met elkaar te communiceren. Ze faciliteren de overdracht van gegevens tussen systemen. Er zijn verschillende API protocollen die mogelijk gebruikt kunnen worden
en het is vooral de toepassing van de applicatie zelf die zal helpen een keuze te maken. Zodra de vereisten gekend zijn, kan gekeken worden welke technologie hier
het meeste aan bijdraagt. Om een goede keuze te maken moeten uiteraard de verschillen tussen de mogelijke API's gekend zijn en ook in welke scenario's welke technologie voordelen biedt.\\

In dit onderzoek wordt een Representational state transfer, REST, API vergeleken met een Remote Procedure Call, RPC, API, meer specifiek de
implementatie van Google nl. gRPC. Een REST API maakt gebruik van HTTP-protocollen en de uniform resource identifier, URI, om opdrachten en gegevens uit te wisselen.
Het gebruikte HTTP-protocol en URI geven de gebruiker reeds veel informatie over het verwachte gedrag.
RPC API's bieden functies aan die door gebruikers kunnen aangeroepen worden.
Buiten de naam van de functie of methode is er bij het aanroepen geen extra informatie beschikbaar.
Google heeft met gRPC haar eigen implementatie van RPC gemaakt waarbij via specifieke bestanden een contract wordt gedeeld waarmee informatie kan uitgewisseld worden.\\

Het scenario waar dit onderzoek zich op toespitst, is het performantieverschil tussen gRPC en REST bij het versturen van datasets van variërende grootte tussen 2 applicaties.
Softwareontwikkelaars of -architecten die beide technologie\"en overwegen kunnen in de bevindingen van dit onderzoek mogelijk extra inzichten treffen
om, in vergelijkbare scenario's, beter onderbouwd een keuze te maken.\\

In Hoofdstuk~\ref{ch:stand-van-zaken}, wordt een overzicht gegeven van de stand van zaken binnen het onderzoeksdomein op basis van een literatuurstudie.
Deze zal eerst meer duidelijkheid scheppen over wat een API juist is en het belang ervan. De precieze werking van REST alsook van gRPC
worden dan van naderbij bekeken. Dit met de nodige aandacht voor eventuele bijkomende factoren die de performantie kunnen beïnvloeden.
Tot slot worden beide technologieën ook met elkaar vergeleken.\\

In Hoofdstuk~\ref{ch:methodologie} wordt de methodologie toegelicht, aan de hand van de vergelijkende studie in Hoofdstuk~\ref{ch:stand-van-zaken}, en worden de daarin gemaakte keuzes onderbouwd.
Deze hebben o.m. betrekking op de communicerende applicaties, het datatype dat wordt verzonden, de wijze waarop de data wordt verzonden
en welke grootte van datasets beschouwd zullen worden.\\

In de eerste sectie van Hoofdstuk~\ref{ch:conclusie} worden eerst de bevindingen van het onderzoek weergegeven en nadien
volgt de conclusie met daarin een antwoord op de onderzoeksvragen.
Tot slot wordt ook een aanzet gegeven voor toekomstig onderzoek binnen dit domein.