%%=============================================================================
%% Methodologie
%%=============================================================================

\chapter{\IfLanguageName{dutch}{Methodologie}{Methodology}}%
\label{ch:methodologie}

%% TODO: Hoe ben je te werk gegaan? Verdeel je onderzoek in grote fasen, en
%% licht in elke fase toe welke stappen je gevolgd hebt. Verantwoord waarom je
%% op deze manier te werk gegaan bent. Je moet kunnen aantonen dat je de best
%% mogelijke manier toegepast hebt om een antwoord te vinden op de
%% onderzoeksvraag.

\section{Inleiding}

Voor het analyseren van het performantieverschil tussen REST en gRPC is er voor beide technologiën een server nodig die een api aanbiedt dat geconsumeerd kan worden en
een client die deze server aanroept. De tijdsduur tussen de start en einde van de request, nl. het aanroepen van het api door de client
tot het moment waarop alle gevraagde data bij de client toegekomen is,vormt het belangrijkste vergelijkingspunt.
Naast de duurtijd wordt er gekeken worden naar de grootte van de originele dataset die verstuurd wordt gezien REST en gRPC
een verschillend serialisatieproces hanteren waarbij deze laatste technologie compressie zal toepassen.

\section{Praktisch}

Voor alle stadia van het onderzoek wordt er gebruik gemaakt van een java applicatie die zowel een REST en gRPC api aanbiedt.
In casu zal het gaan om een api's die toelaten een lijst van fictieve personen op te vragen. Het aantal personen dat verzonden wordt moet door de client worden aangegeven.
De eerste keer dat een lijst wordt opgevraagd wordt de data gegenereerd en opgeslagen in het geheugen van de applicatie volgens het aantal gevraagde objecten.
Een persoon instantie bevat 5 variabelen die tijdens het genereren van de data m.b.v.\ RandomStringUtils~\parencite{RandomStringUtils}
en RandomDataGenerator~\parencite{RandomDataGenerator} van Apache Commons opgevuld worden met willekeurige tekst of cijfers naargelang het type van variabele.
Voor elk gevraagd aantal zal de eerste request dus als opwarming dienen en niet bruikbaar zijn als vergelijk tussen beide technologiën.
De java applicatie is ontwikkeld met behulp van het Quarkus framework. Quarkus biedt zowel ondersteuning voor REST als voor gRPC.
Voor de REST api maakt Quarkus gebruik van RESTEasy Reactive, een Jakarta REST implementatie. Jakarta REST is een REST API standaard voor Java.
Het serialisatieproces naar JSON gebeurd met Jackson. Voor het gRPC api wordt gestart vanaf Quarkus' Getting Started With gRPC tutorial~\parencite{quarkusgRPC}
Gradle is tot slot gekozen voor dependency management en het builden van de applicatie.

De broncode van deze applicatie is publiek beschikbaar op Github via https://github.com/thiersnicolas/gRPC_vs_REST.git

%%?Deploy to openshift
%%?TLS

%%?postman = geen deserialisatie, best voor fase 1 ook client gebruiken!

~\autocite{quarkusAbout}\\
~\autocite{quarkusREST}\\
~\autocite{Jakarta}\\
~\autocite{quarkusgRPC}\\
~\autocite{Gradle}\\

\section{Enkelvoudige requests}

Er wordt aangevangen door via REST en gRPC requests te versturen voor datasets van variërende grootte. Bij REST wordt alles steeds in zijn geheel verzonden,
voor gRPC worden er echter 2 methods voorzien, één waarbij de lijst in zijn geheel wordt verzonden en een tweede waarbij dit als stream gebeurd.
Deze requests zullen verzonden worden met de applicatie Postman. Deze applicatie laat toe REST en gRPC api's aan te spreken en geeft informatie over het verloop van de request.

~\autocite{Postman}\\

\section{Meervoudige requests}

In deze volgende onderzoeksfase wordt er meer rekening gehouden met eventuele praktische beperkingen waarmee applicatie kunnen en zullen geconfronteerd worden.
Zo zal het beschikbare geheugen van een applicatie alsook het aantal simultane requests die mogelijks zullen ontvangen worden het noodzakelijk maken om
de toegelaten reponse grootte te beperken. Voor REST wordt er in dergelijke gevallen vaak voor het pagineren van datasets geopteerd.
Hierbij is de verstuurde dataset slechts een deel van het geheel. De client geeft bij een gepagineerde request aan hoe groot de subset moet zijn
en het hoeveelste deel deze subset is van de gehele dataset. De server geeft de gevraagde subset aangevuld door de subset grootte, de positie t.o.v. de gehele set
en tot slot ook de grootte van de gehele lijst. Praktisch wordt hiervoor het REST endpoint simpelweg meerdere keren na elkaar aangeroepen.
Bij het gRPC api wordt weer gebruik gemaakt van beide bovenvermelde methods. De dataset wordt, zoals bij REST, in delen verzonden en tevens in zijn geheel als stream verzonden.
Voor deze face word een client applicatie gebruikt voor de verschillende iteraties uit te voeren. De tijdregistratie gebeurd door middel van Quarkus Micrometer Metrics.

~\autocite{quarkusMicrometer}\\

\section{Requests voorafgegaan en/of gevolgd door andere processen}

In deze derde en laatste onderzoeksfase worden verschillende scenario's vergeleken waarbij andere processen dan de communicatie tussen server en client
ook een rol spelen. \newline
Het eerste scenario voorziet dat de dataset, serverside, ditmaal niet in het geheugen beschikbaar is, maar eerst uit een mongo~\parencite{MongoDB} database collectie moet worden opgehaald.
Voor REST zijn niet gepagineerde requests en gepagineerde requests voor grote datasets voorzien. Bij het gRPC api worden ook weer beide methods gebruikt maar bij de stream versie
wordt de stream vanuit mongo rechtstreeks doorgegeven. \newline
Voor het tweede scenario wordt de dataset uit het geheugen verzonden, maar wordt aan de zijde van de client de response gedeserialiseerd en weggeschreven naar een bestand.
Tot slot worden beide voorgaande processen samengevoegd en zal de dataset in de server uit de mongo gehaald worden en dan, clientside, naar een bestand worden weggeschreven.

Voor de performantie van REST te vergelijken met gRPC zal er een java applicatie gemaakt worden die een REST en een gRPC api aanbiedt.
In casu zullen beide api's aangeroepen kunnen worden om een lijst personen op te vragen waarbij het aantal personen door de client aangegeven dient te worden.
De datasets zullen voor beide api's, en voor hetzelfde opgevraagde aantal, gelijk zijn. Deze server applicatie wordt opgezet met behulp van het Quarkus framework
waarna ze op OpenShift van Red Hat zal gedeployed worden. Quarkus maakt standaard gebruik van RESTEasy reactive
Bei

REST lijkt sneller dan gRPC wanneer er 1 (grote) hoeveelheid data wordt verzonden.
Dit is de ideale wereld. Wanneer er beperkingen zijn van resources zoals memory van de applicatie en multiple calls kan het zijn dat die response
grootte dient beperkt te worden. Bij gRPC gaat het mogelijk zijn om de hele response via eenzelfde stream door te geven terwijl REST meerdere calls zal nodig hebben (to test!).

